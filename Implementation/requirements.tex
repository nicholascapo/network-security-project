\section{Requirements Analysis}

\subsection{Language and Binary}
\begin{enumerate}
\item The program shall be referred to herein as \emph{Viper}
\item One version of the program shall be produced using the C language
\item One version of the program shall be produced using the Python language
\item Each version shall be compiled into two binaries (\texttt{viper} and \texttt{viper-test}) with the following usage:
	\begin{enumerate}
	\item \texttt{viper [ -h | --help ]  [ -e | -d | --encrypt | --decrypt ] [ -t | --threads NUM ] [ -k | --key KEY ]}
	\item \texttt{viper-test [ -h | --help ]  [ -e | -d | --encrypt | --decrypt ] [ -k | --key KEY ] input$\_$block}
	\end{enumerate}
\end{enumerate}

\subsection{Input/Output}
\begin{enumerate}
\item \texttt{viper} shall expect input on \texttt{stdin}, and generate output on \texttt{stdout}
\item \texttt{viper-test} shall expect a single block of 32 hexadecimal values as the last argument on the command line
\item \texttt{viper} shall be the general case of \texttt{viper-test} and shall encrypt or decrypt until reaching end--of-input
\item All errors and help texts shall be written to \texttt{stderr}
\end{enumerate}

\subsection{Compatibility}
\begin{enumerate}
\item Each version of \texttt{viper} shall be ciphertext compatible with the reference implementation of \texttt{Serpent}
\end{enumerate}

\subsection{Threading}
\begin{enumerate}
\item Each version of \texttt{viper} shall implement a single threaded \textbf{TODO}
\item Each version of \texttt{viper} shall implement a multi-threaded \textbf{TODO}
\end{enumerate}


